During the 1920s, Johns Hopkins University professors Lowell Reed and Hampton Wade Frost developed and introduced to their university classes what we now know as the Reed-Frost Model. Unpublished in its time due to Frost’s belief that it was insignificant \cite{dietz}, it was first formally analyzed in a television broadcast by Johns Hopkins in 1951 \cite{reed} and also by Helen Abbey in her 1952 journal article \emph{An Examination of the Reed-Frost Theory of Epidemics} \cite{abbey}.

Reed and Frosts’ model was an adaptation from one created by Sopher in 1927, with the modification that only one new case would result if an uninfected (or susceptible) person came into contact with two different sick (infectious) people \cite{abbey}. The model is an example of an SIR model (susceptible, infected, and recovered/removed), where there are three discrete states that a person in the population will fall into. The first state is comprised of individuals who have yet to be exposed to the illness in question and have no immunity to it, in other words, they are \emph{susceptible} to infection. State two is for those people who are currently ill and can transmit the disease to susceptible individuals. We will refer to the last state as ‘recovered’, meaning those individuals who have contracted the illness and are no longer contagious and cannot become ill again. This last state can also be thought of as the ‘removed’ state since it will be made up of those individuals who are no longer susceptible by way of immunity or death from the disease \cite{mark}. 

At the start of modeling, the majority of the population will be in the susceptible group, except for the initial individuals who are infectious. At each time step any susceptible individual who has been exposed will move to the infectious group and any individual who has recovered will move to the recovered group. The model is simple in the fact that it assumes there is no incubation period to delay the onset of infectiousness after exposure, that there is no asymptomatic spread, and that once a person recovers, they cannot become susceptible again \cite{halloran}.

In her article, Helen Abbey goes through an examination of the Reed-Frost Model and applies it to data from the Medical Research Council Special Reports, \emph{Epidemics in Schools} which contains information on disease occurrence in English Naval and boarding schools \cite{abbey}. She notes that while there is a lot of existing epidemiologic data, much of it lacks the necessary qualities and fails to meet some of the specific assumptions required of the Reed-Frost Model, which will be discussed in the third section. However, the Reed-Frost Model remains a strong, yet somewhat simplistic, model for understanding epidemics. 