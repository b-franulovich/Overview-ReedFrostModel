The Reed-Frost Model has been in existence for almost 100 years and is a foundational model in the modeling of epidemics. In this paper we explored the history of the model, the two versions of it, and examined how to estimate p given a dataset that meets the assumptions. As a type of SIR model, it has three discrete states in which individuals in a closed population fall into: susceptible, infectious, and recovered. 

The two forms of the model are deterministic and stochastic, with the stochastic model being that the number of infectious individuals has a binomial probability distribution based on the probability that a susceptible individual will become infected in time $t$. Each version of the model allows the number of individuals in each state at time $t$ to be calculated or simulated (in the stochastic model). To estimate $p$ numerical maximization must be employed to first find $\hat{q}$ and then used to find $\hat{p}$. 

We have seen from Helen Abbey's paper that the model can be useful in modeling closed population epidemics, but can be challenging due to its strict assumptions. However, despite these limitations, the model will always be a significant contribution to the field of epidemiology.
